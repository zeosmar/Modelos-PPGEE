\chapter{Introduction} \label{ch:intro}
A \gls{wami} System \cite{Blasch2014} provides a big-picture of large areas of interest for remote sensing applications, which increases situation awareness, facilitates law-enforcement, tracking and mapping wildfires and border controls. This system also assists in strategic engagement and police enforcement by monitoring crowd behavior and tracking targets. However, tracking and monitoring algorithms are impaired by the presence of shadows (which modify the color and form of objects) in the images acquired by the system. We present in this work a method to automatically detect and remove shadows in images captured by the \gls{wami} System with local, on-line, real-time image processing.

A visual system, be it natural or artificial, is only complete when all information about a scene is obtained. This additional information provides clues about different environmental conditions, such as the effects of optical perspectives, object overlaps and shadows. In remote sensing applications, shadows are often considered as nuisances, especially at low resolutions. They are known to modify the form and color of the \glspl{roi} \cite{pan2014}, \emph{e.g.} a green object may appear black if there is low luminance. The shadow existence may also lead to misidentification within regions where water bodies or vegetation are present due to the similarity of their spectral signatures. Furthermore, images captured from airspace are subject to change depending on the position of the light source and the movements performed by the acquisition hardware.

% checked - Guilherme
Most of the techniques applied in computer vision are based in the human visual experience. The human visual system relies not only in the rod and cone cells, but in the preprocessing of the image done by the retina and further intervention of the brain. One of the characteristics inspired by the human vision is that of the photometric invariance. Humans perceive color as approximately constant irrespective of the illuminant, provided a sufficient amount of light is available. We describe colors in terms of hue, saturation and brightness, perceptions that several color spaces such as the \gls{hsl}, \gls{hsi}, and \gls{hsv} color spaces try to mimic.

%
In color aerial images, color tone is a powerful descriptor that simplifies and dominates the identifying characteristics of visual interpretation applications, so \gls{hsl}-like color spaces are used in computer vision to separate chromaticity from luminance \cite{gonzalez1992}, which facilitates object detection regardless if it is under shade or not. However, the amount of different \gls{hsl} type color spaces across the literature and their device-dependence has caused many authors providing different equations for the same color space \cite{ford1998colour}. Although being very good for user interfaces, specially color selection, those \gls{hsl} related color spaces provide a mere approximation of the luminance information in the image and often confound saturation and lightness or hue and lightness \cite{brewer1994color}, with the same colors being formed as a result of different combinations of those. True lightness calculation requires an appropriate color space such as \acrshort{cielab}, or its polar counterpart \acrshort{cielch}, both defined by the \glslink{cie}{\textit{Commission Internationale de l'Eclairage} (CIE)} \cite{ford1998colour}.

% checked - Guilherme
This dissertation presents a method that is intended to be included into the \gls{wami} System mentioned above, which focus on aerial imaging of crowd behavior, traffic analysis, border controls, agricultural monitoring or any field/application that would benefit from a bigger picture of a ROI. One of the main attractions of this system is the ability to track objects automatically. However, this algorithm is being impacted by the presence of shadows. If an object, say a car, goes under shade and then back to an unshaded area, the tracking system may lose track of the object, due to the momentary alteration to its form, color and intensity. This reveals the main motivation of this dissertation, to restore the appearance of an object under shade to its original appearance under sunlight.

% remove - Guilherme
%Designed by Transparent Sky\texttrademark, the system used in this work is composed of an embedded computer with a sensor imaging earth from an airplane around 2000 meters high. The images acquired by the system have 29MP and cover an area of 2.5 square miles with Ground Sampling Distance (GDS) of about to 30cm. Images are geo-referenced (overlapped with satellite earth images) and stabilized by software. The image processing is done in parallel, using an Nvidia GPU, which provides near real-time processing so the throughput of the processed images is faster than the rate of which the sensor captures images, 4 images per second. Recently, a drone version of the system is also available.\par

\section{Objectives} \label{sec:objectives}
% checked - Guilherme
The main objective of this dissertation is to detect shadows of an aerial image and remove them from the original image, while providing an implementation capable of real-time processing. The final outcome of the software is an image, with no shadows, that maintains the original color and form of objects.

% checked - Guilherme
We defined the following partial objectives as to keep track of the progress of this study:
\begin{enumerate}
\item Convert the original image to a color space that decouples color information from luminance information, approximating the human eye response.
\item Modify the Specthem Ratio in order to better enhance shaded areas.
\item Find a segmentation method that meets the application requirements.
\item Develop a method to relight shaded regions to restore object colors and forms.
\item Implement the method for execution in the GPU, aiming real-time processing so the throughput of processed images is faster than the image acquisition rate.
\end{enumerate}

\section{Structure}
% MUST REVISE AFTER EVERYTHING
This work is organized as follows. In the next chapter, we go over the background to better situate the concepts of this dissertation. We describe aerial photogrammetry, the human visual system, photometric invariance and shadow characteristics. We also include the state of the art in shadow detection and shadow removal. Our method is extensively described in two different chapters, with Chapter \ref{ch:shadow_detection} covering shadow detection and Chapter \ref{ch:shadow_removal} covering shadow removal. Chapter \ref{ch:method} describes the materials and methods that were used as in hardware, software, programming libraries and \gls{gpu} programming. This chapter also describes the \gls{wami} System which inspired this work, in Section \ref{sec:wami}. We discuss assessment techniques in Chapter \ref{ch:assess}. Our results are presented in Chapter \ref{ch:results}, for both shadow detection and shadow removal, comparing this and other methods in the literature, regarding both correctness and precision as well as execution time. Finally, we make our final considerations in Chapter \ref{ch:conclusions}.
